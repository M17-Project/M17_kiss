% if you need a document suitable for printing, remove the ",oneside" from the following line.
\documentclass[a4paper,11pt,oneside]{article}
\usepackage[centering,margin=2.5cm]{geometry}
\usepackage[export]{adjustbox}
\usepackage[utf8]{inputenc}
\usepackage[T1]{fontenc}
\usepackage{PTSerif}
\usepackage{parskip}
\usepackage{tabularx}
\usepackage{tabularray}
\usepackage{multirow}
\usepackage{float}
\usepackage{tikz}
\usetikzlibrary{shapes.geometric, arrows.meta}
\tikzstyle{textonly} = [rectangle,
minimum width=1cm,
minimum height=1cm,
text centered,
draw=white]
\tikzstyle{whrectround} = [rectangle, rounded corners,
minimum width=1cm,
minimum height=1cm,
text centered,
text width=2cm,
draw=black]
\usepackage{amstext}
\usepackage{array,calc}
\newcolumntype{L}{>{$}l<{$}}
\usepackage{graphicx}
\renewcommand{\arraystretch}{1.5}
\usepackage{listings}
\usepackage{xcolor}
\definecolor{codegreen}{rgb}{0,0.6,0}
\definecolor{codegray}{rgb}{0.5,0.5,0.5}
\definecolor{codepurple}{rgb}{0.58,0,0.82}
\definecolor{backcolour}{rgb}{0.95,0.95,0.92}
\lstdefinestyle{codestyle}{
	backgroundcolor=\color{backcolour},
	commentstyle=\color{codegreen},
	keywordstyle=\color{magenta},
	numberstyle=\tiny\color{codegray},
	stringstyle=\color{codepurple},
	basicstyle=\ttfamily\footnotesize,
	breakatwhitespace=false,
	breaklines=true,
	captionpos=b,
	keepspaces=true,
	numbers=left,
	numbersep=5pt,
	showspaces=false,
	showstringspaces=false,
	showtabs=false,
	tabsize=2
}
\lstset{style=codestyle}
\usepackage{amsmath}
\setcounter{MaxMatrixCols}{20}
\usepackage{longtable,booktabs}
\makeatletter
% \renewcommand{\frontmatter}{\cleardoublepage\@mainmatterfalse}
% \renewcommand{\mainmatter}{\cleardoublepage\@mainmattertrue}
\makeatother
\usepackage[pdftex,
			pdfauthor={Wojciech Kaczmarski SP5WWP et al.},
			pdftitle={M17 KISS},
			pdfsubject={KISS for M17},
			pdfkeywords={m17, amateur radio, ham radio, digital, digital radio, codec 2, open source, specification},
			]{hyperref}

%opening
\title{M17 KISS}
\author{Wojciech Kaczmarski SP5WWP et al.}

\begin{document}

\begin{titlepage}
	\raggedleft
	\includegraphics[width=0.7\linewidth,right]{img/m17_logo_shadow}
	\vspace*{\baselineskip}
	{\Large Wojciech Kaczmarski SP5WWP et al.} \\
	\vspace*{0.167\textheight}
	\textbf{\LARGE M17 KISS Protocol} \\
	\today
	\vfill
	{\large Version 0.1}
	\vfill
	\LaTeX version compiled by Steve Miller KC1AWV
\end{titlepage}

\tableofcontents

\pagebreak

\paragraph{M17 KISS}

Copyright \copyright{}  2023-2025  M17 Project. \\

Permission is granted to copy, distribute and/or modify this document under the terms of the GNU Free Documentation License, Version 1.3 or any later version published by the Free Software Foundation; with no Invariant Sections, no Front-Cover Texts, and no Back-Cover Texts. A copy of the license is included in the section entitled ``GNU Free Documentation License'' or at the following web page: \href{https://www.gnu.org/licenses/fdl-1.3.en.html}{https://www.gnu.org/licenses/fdl-1.3.en.html}

\pagebreak

\section{M17 KISS}

This document discusses the conventions for adapting KISS TNCs to M17 packet and streaming modes. M17 is a more complex protocol, both at the baseband level and at the data link layer than is typical for HDLC-based protocols commonly used on KISS TNCs. However, it is well suited for modern packet data links, and can even be used to stream digital audio between a host and a radio.

This document assumes the reader is familiar with the streaming and packet modes defined in the M17 spec, and with KISS TNCs and the KISS protocol.

In all cases, the TNC expects to get the data payload to be sent and is responsible for frame construction, FEC encoding, puncturing, interleaving and decorrelation. It is also responsible for baseband modulation.

For streaming modes, all voice encoding (Codec2) is done on the host and not on the TNC. The host is also responsible for constructing the LICH.

\section{Glossary}

\subsection{TNC}

Terminal node controller -- a baseband network interface device to allow host computers to send data over a radio network, similar to a modem. It connects a computer to a radio and handles the baseband portion of the physical layer and the data link layer of network protocol stack.

\subsection{KISS}

Short for ``Keep it simple, stupid''. A simplified TNC protocol designed to move everything except for the physical layer and the data link layer out of the TNC. Early TNCs could include everything up through the application layer of the OSI network model.

\subsection{SLIP}

\href{https://en.wikipedia.org/wiki/Serial_Line_Internet_Protocol}{Serial Line Internet Protocol} -- the base protocol used by the KISS protocol, extended by adding a single \textbf{type indicator} byte at the start of a frame.

\subsection{type indicator}

A one byte code at the beginning of a KISS frame which indicates the TNC \textbf{port} and KISS \textbf{command}.

\subsection{port}

A logical port on a TNC. This allowed a single TNC to connect to multiple radios. Its specific use is loosely defined in the KISS spec. The high nibble of the KISS \textbf{type indicator}. Port 0xF is reserved.

\subsection{command}

A KISS command. This tells the TNC or host how to interpret the KISS frame contents. The low nibble of the KISS \textbf{type indicator}. Command 0xF is reserved.

\subsection{CSMA}

\href{https://en.wikipedia.org/wiki/Carrier-sense_multiple_access}{Carrier-sense multiple access} -- a protocol used by network devices to minimize collisions on a shared communications channel.

\subsection{HDLC}

\href{https://en.wikipedia.org/wiki/High-Level_Data_Link_Control}{High-Level Data Link Control} -- a data link layer framing protocol used in many AX.25 packet radio networks. Many existing protocol documents, including KISS, reference HDLC because of its ubiquity when the protocols were invented. However, HDLC is not a requirement for higher level protocols like KISS which are agnostic to the framing used at the data link layer.

\subsection{EOS}

End of stream -- an indicator bit in the frame number field of a stream data frame.

\subsection{LICH}

Link information channel -- a secondary data channel in the stream data frame containing supplemental information, including a copy of the link setup frame.

\section{M17 Protocols}

This specification defines KISS TNC modes for M17 packet and streaming modes, allowing the KISS protocol to be used to send and receive M17 packet and voice data. Both are bidirectional. There are two packet modes defined. This is done to provide complete access to the M17 protocol while maintaining the greatest degree of backwards compatibility with existing packet applications.

These protocols map to specific KISS port. The host tells the TNC what type of data to transmit based on the port used in host to TNC transfers. And the TNC tells the host what data it has received by the port set on TNC to host transfers.

This document outlines first the two packet protocols, followed by the streaming protocol.

\section{KISS Basics}

\subsection{TX Delay}

If a \textbf{KISS TX} delay $T_d$ greater than 0 is specified, the transmitter is keyed for $T_d \times 10ms$ with only a DC signal present.

The $T_d$ value should be adjusted to the minimum required by the
transmitter in order to transmit the full preamble reliably.

Only a single 40ms preamble frame is ever sent.

\begin{quote}
	NOTE: A TX delay may be necessary because many radios require some 	time between when PTT is engaged and the transmitter can begin 	transmitting a modulated signal.
\end{quote}

\section{Packet Protocols}

In order to provide backward compatibility with the widest range of existing ham radio software, and to make use of features in the the M17 protocol itself, we will define two distinct packet interfaces BASIC and FULL.

The KISS protocol allows us to target specific modems using the port identifier in the control byte.

We first define basic packet mode as this is initially likely to be the most commonly used mode over KISS.

\subsection{M17 Basic Packet Mode}

Basic packet mode uses only the standard KISS protocol on TNC port 0. This is the default port for all TNCs. Packets are sent using command 0. Again, this is normal behavior for KISS client applications.

\paragraph{Sending Data}

In basic mode, the TNC only expects to receive packets from the host, as it would for any other mode supported AFSK, G3RUH, etc.

If the TNC is configured for half-duplex, the TNC will do P-persistence CSMA using a 40ms slot time and obey the P value set via the KISS interface. CSMA is disabled in full-duplex mode.

The \textbf{TX Tail} value is deprecated and is ignored.

The TNC sends the preamble burst.

The TNC is responsible for constructing the link setup frame, identifying the content as a raw mode packet. The source field is an encoded TNC identifier, similar to the APRS TOCALL, but it can be an arbitrary text string up to 9 characters in length. The destination is set to the broadcast address.

In basic packet mode, it is expected that the sender callsign is embedded within the packet payload.

The TNC sends the link setup frame.

The TNC then computes the CRC for the full packet, splits the packet into data frames encode and modulate each frame back-to-back until the packet is completely transmitted.

If there is another packet to be sent, the preamble can be skipped and the TNC will construct the next link setup frame (it can re-use the same link setup frame as it does not change) and send the next set of packet frames.

\paragraph{Limitations}

The KISS specification defines no limitation to the packet size allowed. Nor does it specify any means of returning error conditions back to the host. M17 packet protocol limits the raw packet payload size to 823 bytes. The TNC must drop any packets larger than this.

\paragraph{Receiving Data}

When receiving M17 data, the TNC must receive and parse the link setup frame and verify that the following frames contain raw packet data.

The TNC is responsible for decoding each packet, assembling the packet from the sequence of frames received, and verifying the packet checksum. If the checksum is valid, the TNC transfers the packet, excluding the CRC to the host using \textbf{KISS port} 0.

\subsection{M17 Full Packet Mode}

The purpose of full packet mode is to provide access to the entire M17 packet protocol to the host. This allows the host to set the source and destination fields, filter received packets based on the content these fields, enable encryption, and send and receive type-coded frames.

Use M17 full packet mode by sending to \textbf{KISS port} 1. In this mode the host is responsible for sending both the link setup frame and the packet data. It does this by prepending the 30-byte link setup frame to the packet data, sending this to the TNC in a single KISS frame. The TNC uses the first 30 bytes as the link setup frame verbatim, then splits the remaining data into M17 packet frames.

As with basic mode, the TNC uses the \textbf{Duplex} setting to enable/disable CSMA, and uses the \textbf{P value} for CSMA, with a fixed slot time of ``4'' (40 ms).

\paragraph{Receiving Data}

For TNC to host transfers, the same occurs. The TNC combines the link setup frame with the packet frame and sends both in one KISS frame to the host using \textbf{KISS port} 1.

\section{Stream Protocol}

The streaming protocol is fairly trivial to describe. It is used by sending first a link setup frame followed by a stream of 26-byte data frames to KISS port 2.

\subsection{Stream Format}

\paragraph{M17 KISS Stream Protocol}

\begin{table}[H]
	\centering
	\begin{tblr}{
		colspec={lX},
		}
		\hline
		Frame Size & Contents \\
		\hline
		30 & Link Setup Frame \\
		26 & LICH + Payload \\
		26 & LICH + Payload \\
		\ldots{} & \ldots{} \\
		26 & LICH + Payload with EOS bit set \\
		\hline[2px]
	\end{tblr}
	\caption{KISS Stream}
\end{table}

The host must not send any frame to any other KISS port while a stream is active (a frame with the EOS bit has not been sent).

It is a protocol violation to send anything other than a link setup frame with the stream mode bit set in the first field as the first frame in a stream transfer to KISS port 2. Any such frame is ignored.

It is a protocol violation to send anything to any other KISS port while a stream is active. If that happens the stream is terminated and the packet that caused the protocol violation is dropped.

\subsection{Data Frames}

The data frames contain a 6-byte (48-bit) LICH segment followed by a 20 byte payload segment consisting of frame number, 16-byte data payload and CRC. The TNC is responsible for parsing the frame number and detecting the end-of- stream bit to stop transmitting.

\paragraph{KISS Stream Data Frame}

\begin{table}[H]
	\centering
	\begin{tblr}{
		colspec={lX},
		}
		\hline
		Frame Size & Contents \\
		\hline
		6 & LICH (48 bits) \\
		2 & Frame Number and EOS Flag \\
		16 & Payload \\
		2 & M17 CRC of frame number and payload \\
		\hline[2px]
	\end{tblr}
	\caption{KISS Stream Data}
\end{table}

The TNC is responsible for FEC-encoding both the LICH the payload, as well as interleaving, decorrelation, and baseband modulation.

\subsection{Timing Constraints}

Streaming mode provides additional timing constraints on both host to TNC transfers and on TNC to host transfers. Payload frames must arrive every 40ms and must have a jitter below 40ms. In general, it is expected that the TNC has up to 2 frames buffered (buffering occurs while sending the preamble and link setup frames), it should be able to keep the transmit buffers filled with packet jitter of 40ms.

The TNC must stop transmitting if the transmit buffers are empty. The TNC communicates that it has stopped transmitting early (before seeing a frame with the end of stream indicator set) by sending an empty data frame to the host.

\section{TNC to Host Transfers}

TNC to host transfers are similar in that the TNC first sends the 30-byte link setup frame received to the host, followed by a stream of 26-byte data frames as described above. These are sent using \textbf{KISS port} 2.

The TNC must send the link setup frame first. This means that the TNC must be able to decode LICH segments and assemble a valid link setup frame before it sends the first data frame. The TNC will only send a link setup frame with a valid CRC to the host. After the link setup frame is sent, the TNC ignores the CRC and sends all valid frames (those received after a valid sync word) to the host. If the stream is lost before seeing an end-of-stream flag, the TNC sends a 0-byte data frame to indicate loss of signal.

The TNC must then re-acquire the signal by decoding a valid link setup frame from the LICH in order to resume sending to the host.

\section{Busy Channel Lockout}

The TNC implements \textbf{busy channel lockout} by enabling half-duplex mode on the TNC, and disables \textbf{busy channel lockout} by enabling full-duplex mode. When busy channel lockout occurs, the TNC keeps the link setup frame and discards all data frames until the channel is available. It then sends the preamble, link setup frame, and starts sending the data frames as they are received.

\begin{quote}
	NOTE: BCL will be apparent to a receiver as the first frame received after the link setup frame will not start with frame number 0.
\end{quote}

\subsection{Limitations}

Information is lost by having the TNC decode the LICH. It is not possible to communicate to the host that the LICH bytes are known to be invalid.

Should we have the TNC signal the host by dropping known invalid LICH segments? The host can tell that the LICH is missing by looking at the frame size.

\section{Mixing Modes}

An M17 KISS TNC need not keep track of state across distinct TNC ports. Packet transfers are sent one packet at a time. It is OK to send to port 0 and port 1 in subsequent transfers. It is also OK to send a packet followed immediately by a voice streams. As mentioned earlier, it is a protocol violation to sent a KISS frame to any other port while a stream is active. However, a packet can be sent immediately following a voice stream (after EOS is sent).

\subsection{Back-to-back Transfers}

The TNC is expected to detect back-to-back transfers from the host, even across different KISS ports, and suppress the generation of the preamble.

For example, a packet containing APRS data sent immediately on PTT key-up should be sent immediately after the EOS frame.

Back-to-back transfers are common for packet communication where the window size determines the number of unacknowledged frames which may be outstanding (unacknowledged). Packet applications will frequently send back-to-back packets (up to window size packets) before waiting for the remote end to send ACKs for each of the packets.

\section{Implementation Details}

\subsection{Polarity}

One of the issues that must be addressed by the TNC designer, and one which the KISS protocol offers no ready solution for, is the issue of polarity.

A TNC must interface with a RF transceiver for a complete M17 physical layer implementation. RF transceivers may have different polarity for their TX and RX paths.

M17 defines that the +3 symbol is transmitted with a +2.4 kHz deviation (2.4 kHz above the carrier). \textbf{Normal polarity} in a transceiver results in a positive voltage driving the frequency higher and a lower voltage driving the frequency lower. \textbf{Reverse polarity} is the opposite. A higher voltage drives the frequency lower.

On the receive side the same issue exists. \textbf{Normal polarity} results in a positive voltage output when the received signal is above the carrier frequency. \textbf{Reverse polarity} results in a positive voltage when the frequency is below the carrier.

Just as with transmitter deviation levels and received signal levels, the polarity of the transmit and receive path must be adjustable on a 4-FSK modem. The way these adjustments are made to the TNC are not addressed by the KISS specification.

\end{document}

\end{document}
